\section{Boolsche Algebra}
\subsection{Grundregeln}
\subsubsection{Kommutativität}
\begin{align*}
    A \land B &= B \land A\\
    A \lor B &= B \lor A
\end{align*}
\subsubsection{Assoziativität}
\begin{align*}
    A \land (B \land C) &= (A \land B) \land C\\
    A \lor (B \lor C) &= (A \lor B) \lor C\\
\end{align*}
\subsubsection{Distributivität}
\begin{align*}
    (\textcolor{blue}{A\,\land}~B)\,\textcolor{red}{\lor}\,(\textcolor{blue}{A\,\land}\,C) &= \textcolor{blue}{A\,\land}\,(B\,\textcolor{red}{\lor}\,C)\\
    (\textcolor{blue}{A\,\lor}~B)\,\textcolor{red}{\land}\,(\textcolor{blue}{A\,\lor}\,C) &= \textcolor{blue}{A\,\lor}\,(B\,\textcolor{red}{\land}\,C)\\
\end{align*}

\begin{flushleft}
    \renewcommand{\arraystretch}{1.5}
    \begin{tabular}{l l l}
        Nicht & $\overline{\overline{A}} = A$ & \\
        \hline
        Null-Th. & $A \lor 0 = A$ & $A \land 0 = 0$\\
        \hline
        Eins-Th. & $A \lor 1 = 1$ & $A \land 1 = A$\\
        \hline
        Idempotenz & $A \lor A = A$ & $A \land A = A$\\
        \hline
        V. Komp. & $A \lor \overline{A} = 1$ & $A \land \overline{A} = 0$\\
        \hline
        Adsorp. & \multicolumn{2}{l}{$A \lor (\overline{A} \land B) = A \lor B$}\\
        & \multicolumn{2}{l}{$A \land (\overline{A} \lor B) = A \land B$}\\
        \hline
        Adsorp. & \multicolumn{2}{l}{$A \lor (A \land B) = A$}\\
        & \multicolumn{2}{l}{$A \land (A \lor B) = A$}\\
        \hline
        Nachbar.G. & \multicolumn{2}{l}{$(A \land B) \lor (\overline{A} \land B) = B$}\\
        & \multicolumn{2}{l}{$(A \lor B) \land (\overline{A} \lor B) = B$}
    \end{tabular}
\end{flushleft}

\subsection{De Morgan}
\begin{flushleft}
    \renewcommand{\arraystretch}{1.5}
    \begin{tabular}{l l}
        1. Regel & $\overline{A \land B} = \overline{A} \lor \overline{B}$\\
        2. Regel & $\overline{A \lor B} = \overline{A} \land \overline{B}$\\
    \end{tabular}
\end{flushleft}
Regeln gelten auch für $n$ verknüpfte Terme.

\subsection{Normalformen}
\begin{center}
    \small
    \renewcommand{\arraystretch}{1.5}
    \begin{tabular}{p{30mm} | p{30mm}}
        \textbf{Minterm} & \textbf{Maxterm}\\
        \hline
        AND-Ausdruck & OR-Ausdruck\\
        Output: 1 & Output: 0\\ 
        $n$ Schaltvar. $\rightarrow$ $2^n$ mögl. Minterme. & $n$ Schaltvar. $\rightarrow$ $2^n$ mögl. Maxterme.\\
        nicht-invertierte Var: 1 & nicht-invertierte Var: 0\\
        invertierte Var: 0 &invertierte Var: 0
    \end{tabular}
\end{center}
\cemph[black]{Kanonisch Normalform}: Alle Terme einer Schaltfunktion; \underline{nicht vereinfacht oder gekürzt}.
\subsubsection{Disjunktive Normalform}
\begin{enumerate}
    \item Identifiziere WT-Zeilen mit Output 1
    \item \textbf{Minterme} für diese Zeilen aufstellen
    \item Minterme mit \textbf{OR} verknüpfen
\end{enumerate}
\subsubsection{Konjunktive Normalform}
\begin{enumerate}
    \item Identifiziere WT-Zeilen mit Output 0
    \item \textbf{Maxterme} für diese Zeilen aufstellen
    \item Maxterme mit \textbf{AND} verknüpfen
\end{enumerate}
\begin{center}
        \begin{tabular}{|c c|c|c|c|}
            \hline
            A & B & Y & Minterme & Maxterme\\
            \hline
            0 & 0 & 1 & $\overline{A} \land \overline{B}$ & \\
            0 & 1 & 0 & & $A \lor \overline{B}$\\
            1 & 0 & 0 & & $\overline{A} \lor B$\\
            1 & 1 & 1 & $A \land B$ & \\
            \hline
        \end{tabular}
\end{center}

\begin{flushleft}
    \renewcommand{\arraystretch}{1.5}
    \begin{tabular}{l l l}
        \textbf{DNF} & $Y = (\overline{A} \land \overline{B}) \lor (A \land B)$ & {\small 1 Mint. erf. $\rightarrow~1$}\\
        \textbf{KNF} & $Y = (A \lor \overline{B}) \land (\overline{A} \lor B)$ & {\small 1 Maxt. erf. $\rightarrow~0$}\\
    \end{tabular}
\end{flushleft}
\subsubsection{NAND/NOR Schaltungen}
Schaltung nur aus:
\begin{itemize}
    \item NAND: DNF $\rightarrow$ $2 \times$  Negieren $\rightarrow$ $1 \times$ De Morgan
    \item NOR: KNF $\rightarrow$ $2 \times$ Negieren $\rightarrow$ $1 \times$ De Morgan\\
    oder: auf jeden Term der DNF De Morgan anwenden.
\end{itemize}
