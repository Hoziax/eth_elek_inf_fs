\section{Karnaugh Diagramme (KVD)}
\begin{center}
    \begin{minipage}{0.45\linewidth}
        \begin{center}
            \begin{tikzpicture}
                \matrix (kv) [
                    matrix of nodes,
                    nodes in empty cells,
                    column sep=-\pgflinewidth, row sep=-\pgflinewidth,
                    nodes = {
                        rectangle, 
                        draw = black,
                        text width = 3mm,
                        text height = 3mm,
                        align = center
                    }
                ]{
                $0$ & $1$ & $X$ &\\
                & & &\\
                & & &\\
                & & &\\
                };
    
                \path[draw, decorate, decoration=brace] (kv-1-1.north west) -- (kv-1-2.north east) node [midway, above] {A};
                \path[draw, decorate, decoration=brace] (kv-1-3.north west) -- (kv-1-4.north east) node [midway, above] {$\overline{\text{A}}$};
                \path[draw, decorate, decoration=brace] (kv-2-1.south west) -- (kv-1-1.north west) node [midway, left] {C};
                \path[draw, decorate, decoration=brace] (kv-4-1.south west) -- (kv-3-1.north west) node [midway, left] {$\overline{\text{C}}$};
                \path[draw, decorate, decoration=brace] (kv-2-4.north east) -- (kv-3-4.south east) node [midway, right] {D};
                \path[draw, decorate, decoration=brace] (kv-1-4.north east) -- (kv-1-4.south east) node [midway, right] {$\overline{\text{D}}$};
                \path[draw, decorate, decoration=brace] (kv-4-4.north east) -- (kv-4-4.south east) node [midway, right] {$\overline{\text{D}}$};
                \path[draw, decorate, decoration=brace] (kv-4-3.south east) -- (kv-4-2.south west) node [midway, below] {$\overline{\text{B}}$};
                \path[draw, decorate, decoration=brace] (kv-4-1.south east) -- (kv-4-1.south west) node [midway, below] {B};
                \path[draw, decorate, decoration=brace] (kv-4-4.south east) -- (kv-4-4.south west) node [midway, below] {B};
            \end{tikzpicture}
        \end{center}
    \end{minipage}
    \hfill
    \begin{minipage}{0.45\linewidth}
        \begin{center}
            \begin{tikzpicture}
                \matrix (kv) [
                    matrix of nodes,
                    nodes in empty cells,
                    column sep=-\pgflinewidth, row sep=-\pgflinewidth,
                    nodes = {
                        rectangle, 
                        draw = black,
                        text width = 3mm,
                        text height = 3mm,
                        align = center
                    }
                ]{
                \node[kvbinhead] {}; & \node[kvbinhead] {00}; & \node[kvbinhead] {01}; & \node[kvbinhead, text = blue] {11}; & \node[kvbinhead] {10};\\
                \node[kvbinhead] {00}; & $0$ & $1$ & $X$ &\\
                \node[kvbinhead] {01}; & & & &\\
                \node[kvbinhead, text = blue] {11}; & & & &\\
                \node[kvbinhead] {01}; & & & &\\
                };
    
                \node[] at ($(kv-1-1.north east) + (-4mm, 1.5mm)$) {\tiny AB};
                \node[] at ($(kv-1-1.south west) + (-0.8mm, 3.5mm)$) {\tiny CD};
                \draw[] ($(kv-1-1.north west) + (-2.2mm, 2.2mm)$) -- ($(kv-1-1.south east) + (-2.4mm, 2.4mm)$);
            \end{tikzpicture}
        \end{center}
    \end{minipage}
\end{center}
{\small Hat das Karnaugh Diagramm 5 Dimensionen, wird die 5te Dimension auf zwei Tabellen aufgeteilt.}
\paragraph{Don't-Care-Zustände} \emph{\textcolor{red}{$X \in \{0,1\}$}}
Redundante, überflüssige oder unmögliche Kombinationen der Eingangsvariablen werden mit einem \emph{$X$} markiert.

\paragraph{Päckchen}
\begin{itemize}
    \item Päckchen immer rechteckig (Ausnahme: über Ecken).
    \item Umfassen möglichst grosse Zweierpotenz.
    \item Dürfen über Ecken und Grenzen hinausgehen und sich überlappen.
\end{itemize}

\begin{center}
    \begin{minipage}{0.48\linewidth}
        \subsubsection{DNF}
        \begin{enumerate}
            \item KVD ausfüllen.
            \item Päckchen mit \emph{$1$} uo $X$.
            \item Vereinfachte Minterme aufstellen.
            \item Minterme mit OR verbinden.
        \end{enumerate}
    \end{minipage}
    \hfill\vline\hfill
    \begin{minipage}{0.48\linewidth}
        \subsubsection{KNF}
        \begin{enumerate}
            \item KVD ausfüllen.
            \item Päckchen mit \emph{$0$} uo $X$.
            \item Vereinfachte Maxterme aufstellen.
            \item Maxterme mit AND verbinden.
        \end{enumerate}
    \end{minipage}
\end{center}

\subsection{Hazard}
Kurzzeitige, unerwünschte Änderung der Signalwerte, die durch Zeitverzögerung der Gatter entstehen.
\begin{center}
    \begin{minipage}{0.38\linewidth}
        \begin{center}
            \begin{tikzpicture}
                \begin{pgfonlayer}{bg}
                    \def\ppad{1.5pt}
                    \matrix (kv) [
                        matrix of nodes,
                        nodes in empty cells,
                        column sep=-\pgflinewidth, row sep=-\pgflinewidth,
                        nodes = {
                            rectangle, 
                            draw = black,
                            text width = 3mm,
                            text height = 3mm,
                            align = center
                        }
                    ]{
                    \node[kvbinhead] {}; & \node[kvbinhead] {00}; & \node[kvbinhead] {01}; & \node[kvbinhead] {11}; & \node[kvbinhead] {10};\\
                    \node[kvbinhead] {00}; & 0 & 1& 1 & 0\\
                    \node[kvbinhead] {01}; & 0 & 1& 1 & 0\\
                    \node[kvbinhead] {11}; & 1 & 1& 0 & 0\\
                    \node[kvbinhead] {01}; & 1 & 1& 0 & 0\\
                    };
        
                    \node[] at ($(kv-1-1.north east) + (-4mm, 1.5mm)$) {\tiny AB};
                    \node[] at ($(kv-1-1.south west) + (-0.8mm, 3.5mm)$) {\tiny CD};
                    \draw[] ($(kv-1-1.north west) + (-2.2mm, 2.2mm)$) -- ($(kv-1-1.south east) + (-2.4mm, 2.4mm)$);
                \end{pgfonlayer}

                \draw[blue] ($(kv-2-3.north west) + (1.5pt, -1.5pt)$) rectangle ($(kv-3-4.south east) + (-1.5pt, 1.5pt)$);
                \draw[blue] ($(kv-4-2.north west) + (1.5pt, -1.5pt)$) rectangle ($(kv-5-3.south east) + (-1.5pt, 1.5pt)$);
                \draw[darkgreen, thick, dashed] ($(kv-2-3.north west) + (1.5pt, -1.5pt)$) rectangle ($(kv-5-3.south east) + (-1.5pt, 1.5pt)$);

                \begin{pgfonlayer}{tl3}
                    \path[draw, <->, > = stealth, red] ($(kv-3-3.center) + (2pt, 0)$) to[in=45, out = -45] ($(kv-4-3.center) + (2pt, 0)$); 
                    \path[draw, <-, > = stealth, red] ($(kv-5-3.center) + (2pt, 0)$) to[in=90, out = -45] ($(kv-5-3.center) + (5pt, -3mm)$);
                    \path[draw, <-, > = stealth, red] ($(kv-2-3.center) + (2pt, 0)$) to[in=270, out = 45] ($(kv-2-3.center) + (5pt, 3mm)$);
                \end{pgfonlayer}
                \begin{pgfonlayer}{tl2}
                    \node[text = red, fill = white, fill opacity = 0.8, text opacity = 1, rounded corners = 4pt] at ($(kv-3-3.south east) + (2mm, 0mm)$) {\tiny Hazard};
                    \node[text = red, fill = white, fill opacity = 0.8, text opacity = 1, rounded corners = 4pt] at ($(kv-5-3.south east) + (2mm, 0mm)$) {\tiny Hazard};
                \end{pgfonlayer}
            \end{tikzpicture}
        \end{center}
    \end{minipage}
    \hfill
    \begin{minipage}{0.55\linewidth}
        \paragraph{Statische Hazards} Stellen im KVD, an denen sich Päckchen orthogonal berühren, aber nicht überlappen.
        \paragraph{Lösung} Berührende \textcolor{blue}{Päckchen} mit zusätzlichen (möglichst grossen) \textcolor{darkgreen}{Päckchen} verbinden.
    \end{minipage}
\end{center}