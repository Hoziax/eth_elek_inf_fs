\section{Zahlensysteme}
\begin{center}
    \begin{minipage}{0.65\linewidth}
        \begin{center}
            \begin{tabular}{c l}
                $D$ & zu berechnende positive Zahl\\
                $R$ & Basis/Radix von $D$\\
                $b_i$ & Koeffizient
            \end{tabular}
        \end{center}
    \end{minipage}
    \hfill
    \begin{minipage}{0.3\linewidth}
        \begin{equation*}
            D = \sum_{-\infty}^{\infty} b_i \cdot R^i
        \end{equation*}
    \end{minipage}
\end{center}
Darstellung $D$ in Basis $R$: $\dots b_2 b_1 b_0 . b_{-1} b_{-2} \dots _{R}$
\begin{flushleft}
    \begin{tabular}{l c l}
        Dezimal & $10$ & $b_i \in \{0, 1, \dots, 9\}$\\
        Dual/Binär & $2$ & $b_i \in \{0, 1\}$\\
        Oktal & $8$ & $b_i \in \{0, 1, \dots, 7\}$\\
        Hexa & $16$ & $b_i \in \{0, 1, \dots, 9, A, B, C, D, E, F\}$\\
    \end{tabular}
\end{flushleft}

\subsection{Umwandlung Zahlensysteme}
\begin{enumerate}
    \item Ganzzahlige Division mit R: $D/R = Q_0 + r_0$.
    \item $$Q_i/R = Q_{i + 1} + r_{i + 1}$$ bis $Q_i = 0$.
    \item Erste Operation gibt MSB, letze Operation gibt LSB {\small (aka. unten nach oben lesen.)}
\end{enumerate}
\subsubsection{Für $1 > D \geq 0$}
\begin{align*}
    D \cdot R &= P_0 \quad K_{-1} = \floor{P_0} \quad a_{-1} = P_0 - K_{-1}\\ 
    a_{-1} \cdot R &= P_{-1} \dots
\end{align*}
$K_i$: Koeffizienten für Zahlensystem. Erste Operation gibt \emph{MSB}, letze Operation gibt \emph{LSB} (aka von oben nach unten lesen).

\subsubsection{Byte}
\begin{center}
    \begin{tabular}{c|c|c|c|c|c|c|c}
        $2^7$ & $2^6$ & $2^5$ & $2^4$ & $2^3$ & $2^2$ & $2^1$ & $2^0$\\
        $128$ & $64$ & $32$ & $16$ & $8$ & $4$ & $2$ & $1$
    \end{tabular}
\end{center}
\begin{center}
    \begin{tabular}{c|c|c|c}
        $2^{-1}$ & $2^{-2}$ & $2^{-3}$ & $2^{-4}$\\
        $0.5$ & $0.25$ & $0.125$ & $0.0625$
    \end{tabular}
\end{center}

\subsubsection{Binär zu Hex}
\begin{center}
    \begin{tabular}{c c||c c||c c||c c}
        $0000$ & $0$ & $0100$ & $4$ & $1000$ & $8$ & $1100$ & $C$\\
        $0001$ & $1$ & $0101$ & $5$ & $1001$ & $9$ & $1101$ & $D$\\
        $0010$ & $2$ & $0110$ & $6$ & $1010$ & $A$ & $1110$ & $E$\\
        $0011$ & $3$ & $0111$ & $7$ & $1011$ & $B$ & $1111$ & $F$\\
    \end{tabular}
\end{center}
\subsection{Zweierkomplement}
\begin{center}
    \begin{tabular}{c c}
        \multicolumn{2}{c}{Sign Bit}\\
        0: positiv & 1: negativ
    \end{tabular}
\end{center}
\subsubsection{Konstruktion}
\begin{enumerate}
    \item Zahl $|Z|$ in Binär $B$ umwandeln.
    \item $B$ bitweise invertieren
    \item $1$ zu LSB addieren (\cemph{!} Übertrag)
    \item Sign Bit hinzufügen (zuvorderst).
\end{enumerate}
Ist die Blocklänge länger als Zahl, vorangehende $0$(-en) miteinbeziehen.
\subsubsection{\twocom Komplement zu Dezimal}
\begin{equation*}
    D_{(10)} = - b_{n-1} \cdot 2^{n-1} + \sum_{i = 0}^{n-2} b_i \cdot 2^i
\end{equation*}
Wertebereich \twocom-Komp. $\left[-2^{n-1}, 2^{n-1} -1\right]$
\subsection{$m$Q$n$}
\begin{equation*}
    D_{(10)} = -b_m \cdot 2^m + \sum_{i = 0}^{m - 1} b_i \cdot 2^i + \sum_{i = 1}^{n} b_i \cdot 2^{-i}
\end{equation*}
$m$: Vorkommabits, $n$: Nachkommabits\\
Sign-Bit muss nur einmal vor dem $m$ codiert werden.

% \subsubsection{Cheatsheet}
% \begin{center}
%     \begin{tabular}{c|c||c|c}
%         $\mathbb{B}$ & $\mathbb{Z}$ &$\mathbb{B}$ & $\mathbb{Z}$\\
%         \hline
%         $0000$ & $0$  & $1000$ & $-8$\\
%         $0001$ & $+1$ & $1001$ & $-7$\\ 
%         $0010$ & $+2$ & $1010$ & $-6$\\
%         $0011$ & $+3$ & $1011$ & $-5$\\ 
%         $0100$ & $+4$ & $1100$ & $-4$\\ 
%         $0101$ & $+5$ & $1101$ & $-3$\\
%         $0110$ & $+6$ & $1110$ & $-2$\\
%         $0111$ & $+7$ & $1111$ & $-1$\\
%     \end{tabular}
% \end{center}
% \begin{center}
%     \begin{tabular}{c|c||c|c}
%         $\mathbb{B}$ & $\mathbb{R}$ &$\mathbb{B}$ & $\mathbb{R}$\\
%         \hline
%         $0.000$ & $0$ & $1.000$ & $-1.0$\\
%         $0.001$ & $+ 0.125$ & $1.001$ & $-0.875$\\
%         $0.010$ & $+0.25$ & $1.010$ & $-0.75$\\
%         $0.011$ & $+0.375$ & $1.011$ & $-0.625$\\
%         $0.100$ & $+0.5$ & $1.100$ & $-0.5$\\
%         $0.101$ & $+0.625$ & $1.101$ & $-0.375$\\
%         $0.110$ & $+0.75$ & $1.110$ & $-0.25$\\
%         $0.111$ & $+0.875$ & $1.111$ & $-0.125$\\
%     \end{tabular}
% \end{center}

\subsection{Binäre Rechenoperationen}
\begin{center}
    \begin{minipage}{0.45\linewidth}
        \subsubsection{Addition}
        Bitweise Addition der Binärzahlen. Leere Slots werden mit $0$ aufgefüllt.
    \end{minipage}
    \hfill
    \begin{minipage}{0.45\linewidth}
        \subsubsection{Subtraktion}
        Addition via \twocom Komp. Übertrag von MSB ignorieren.
    \end{minipage}
\end{center}
\subsubsection{Multiplikation}
\begin{center}
    \begin{minipage}{0.70\linewidth}
        \begin{itemize}
            \item Bitweise Multiplikation des Multiplikanden $a$ mit $b_i$ des Multiplikator.
            \item Sukzessive Multiplikationen werden um ein Bit ($0$) nach links verschoben.
            \item Anzahl Nachkommabits ergibt sich aus der Summe der Anzahl Nachk.bits der Operatoren.
        \end{itemize}
    \end{minipage}
    \hfill
    \begin{minipage}{0.25\linewidth}
        \begin{align*}
            b_0 \cdot a&\\
            +b_1 \cdot a~0\\
            +b_2 \cdot a~0~0&\\
            +b_3 \cdot a~0~0~0&\\
            \hline
            =\text{Sum}&
        \end{align*}
    \end{minipage}
\end{center}
\subsubsection{Division}
\begin{enumerate}
    \item Identifiziere Teil des Divident $>$ Divisor (Unterblock). Für jede Stelle, sodass Divident $<$ Divisor, 0 in Quotient.
    \item Unterblock $-$ Divisor, 1 an Quotient anhängen, Rest behalten.
    \item An das Resultat der Subtraktion Bits des Dividenten anhängen. Wiederholen bis Subtraktion 0 ergibt.
\end{enumerate}